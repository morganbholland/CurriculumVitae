%%%%%%%%%%%%%%%%%%%%%%%%%%%%%%%%%%%%%%%%%
% "ModernCV" CV and Cover Letter
% LaTeX Template
% Version 1.3 (29/10/16)
%
% This template has been downloaded from:
% http://www.LaTeXTemplates.com
%
% Original author:
% Xavier Danaux (xdanaux@gmail.com) with modifications by:
% Vel (vel@latextemplates.com)
%
% License:
% CC BY-NC-SA 3.0 (http://creativecommons.org/licenses/by-nc-sa/3.0/)
%
% Important note:
% This template requires the moderncv.cls and .sty files to be in the same 
% directory as this .tex file. These files provide the resume style and themes 
% used for structuring the document.
%
%%%%%%%%%%%%%%%%%%%%%%%%%%%%%%%%%%%%%%%%%

%----------------------------------------------------------------------------------------
%	PACKAGES AND OTHER DOCUMENT CONFIGURATIONS
%----------------------------------------------------------------------------------------

\documentclass[11pt,letterpaper,sans]{moderncv} % Font sizes: 10, 11, or 12; paper sizes: a4paper, letterpaper, a5paper, legalpaper, executivepaper or landscape; font families: sans or roman

\moderncvstyle{classic} % CV theme - options include: 'casual' (default), 'classic', 'oldstyle' and 'banking'
\moderncvcolor{black} % CV color - options include: 'blue' (default), 'orange', 'green', 'red', 'purple', 'grey' and 'black'

\usepackage{lipsum} % Used for inserting dummy 'Lorem ipsum' text into the template

\usepackage[scale=0.75]{geometry} % Reduce document margins
%\setlength{\hintscolumnwidth}{3cm} % Uncomment to change the width of the dates column
%\setlength{\makecvtitlenamewidth}{10cm} % For the 'classic' style, uncomment to adjust the width of the space allocated to your name

\usepackage[firstyear=2007,lastyear=2021,
marksyr=0.4ex,marksmo=0.2ex
]{moderntimeline}
\tltextstart[base]{\scriptsize}
\tltextend[north]{\scriptsize}
\tldisablemarksyr


%----------------------------------------------------------------------------------------
%	NAME AND CONTACT INFORMATION SECTION
%----------------------------------------------------------------------------------------

\firstname{Morgan} % Your first name
\familyname{Holland} % Your last name

% All information in this block is optional, comment out any lines you don't need
\title{Curriculum Vitae}
\address{2750 Old Saint Augustine Rd.}{Tallahassee, FL 32301}
\mobile{(601) 572 7025}
%\phone{(000) 111 1112}
%\fax{(000) 111 1113}
\email{mbholland@fsu.edu}
\homepage{morganbholland.github.io}{morganbholland.github.io} % The first argument is the url for the clickable link, the second argument is the url displayed in the template - this allows special characters to be displayed such as the tilde in this example
%\extrainfo{additional information}
%\photo[70pt][0.4pt]{pictures/picture} % The first bracket is the picture height, the second is the thickness of the frame around the picture (0pt for no frame)
\quote{I am a PhD candidate at Florida State University focused on Financial Economics, Macroeconomics, and Applied Econometrics}

%----------------------------------------------------------------------------------------

\begin{document}

%----------------------------------------------------------------------------------------
%	COVER LETTER
%----------------------------------------------------------------------------------------

% To remove the cover letter, comment out this entire block

%\clearpage
%
%\recipient{HR Department}{Corporation\\123 Pleasant Lane\\12345 City, State} % Letter recipient
%\date{\today} % Letter date
%\opening{Dear Sir or Madam,} % Opening greeting
%\closing{Sincerely yours,} % Closing phrase
%\enclosure[Attached]{curriculum vit\ae{}} % List of enclosed documents
%
%\makelettertitle % Print letter title
%
%\lipsum[1-2] % Dummy text
%\lipsum[4] % Dummy text
%
%\makeletterclosing % Print letter signature
%
%\newpage

%----------------------------------------------------------------------------------------
%	CURRICULUM VITAE
%----------------------------------------------------------------------------------------

\makecvtitle % Print the CV title

%----------------------------------------------------------------------------------------
%	EDUCATION SECTION
%----------------------------------------------------------------------------------------

\section{Education}

\tlcventry{2015}{0}{Doctorate of Philosophy in Economics}{Florida State University COSSPP}{Tallahassee}{Expected May 2022}{} % Arguments not required can be left empty
\tlcventry{2013}{2015}{Master of Arts in Economics}{The University of South Carolina Moore School of Business}{Columbia}{}{}
\tlcventry{2009}{2010}{Bachelor of Business Administration}{Mississippi College School of Business}{Clinton}{}{}
\tlcventry{2007}{2009}{Associate of Arts in Business Administration}{}{}{}{}

\section{Research}

\subsection{Job Market Paper}
\cventry{2021}{Morgan Holland}{}{\emph{The Role of Long-Term and Short-Term Risk in Relationship Banking}}{``This paper explores the benefits firms may receive from relationship banking that arise from the information gleaned by banks through monitoring. If monitoring reveals not only current output, but also new information about future payoffs, lenders can shield themselves from future losses through early termination of lending agreements. In a competitive lending environment, banks shift the benefits of early termination to borrowers through the lending terms, improving not only the overall expected payoff of projects, but also the welfare of borrowers. Numerical results reveal that the benefits of long-term relationships based on the information revealed in monitoring could be substantial."}{}


\cvitem{Title}{\emph{The Role of Long-Term and Short-Term Risk in Relationship Banking}}{``This paper explores the benefits firms may receive from relationship banking that arise from the information gleaned by banks through monitoring. If monitoring reveals not only current output, but also new information about future payoffs, lenders can shield themselves from future losses through early termination of lending agreements. In a competitive lending environment, banks shift the benefits of early termination to borrowers through the lending terms, improving not only the overall expected payoff of projects, but also the welfare of borrowers. Numerical results reveal that the benefits of long-term relationships based on the information revealed in monitoring could be substantial."}
\cvitem{Abstract}{``This paper explores the benefits firms may receive from relationship banking that arise from the information gleaned by banks through monitoring. If monitoring reveals not only current output, but also new information about future payoffs, lenders can shield themselves from future losses through early termination of lending agreements. In a competitive lending environment, banks shift the benefits of early termination to borrowers through the lending terms, improving not only the overall expected payoff of projects, but also the welfare of borrowers. Numerical results reveal that the benefits of long-term relationships based on the information revealed in monitoring could be substantial."}

\section{Working Papers}
\cvitem{Title}{\emph{The Role of Long-Term and Short-Term Risk in Relationship Banking}}
\cvitem{Abstract}{``We study the distributional consequences of automation in a model with two kinds of agents --- workers, who supply labor, and entrepreneurs, who own capital. We assume that production involves tasks that can be done by either capital or labor with varying productivity. We conceptualize automation as a shift in the relative productivity of capital at certain tasks that reduces the set of tasks done by labor. We contrast this with ``traditional technical progress'', which is an increase in capital productivity at tasks previously done by capital. We derive a simple condition that governs whether labor share goes to zero in the long run, for given tax rates. We then characterize the distributional consequences of a shift in technology, using a tractable case that allows us to cleanly distinguish between automation and traditional technological progress. Finally, we endogenize the tax rate by computing the political economy equilibrium under majority voting, where the government has access to a capital tax and a transfer to workers (a ``universal basic income''). We give conditions for zero or positive capital taxation in the steady state, and conditions under which workers prefer that the labor share go to zero and they derive income wholly from the UBI."}


\cvitem{Job Market Paper}{\emph{The Role of Long-Term and Short-Term Risk in Relationship Banking}}

\cvitem{\textbf{Working Papers}}{
	\begin{itemize}
		\item \emph{The Role of Long-Term and Short-Term Risk in Relationship Banking}. (Job Market Paper)\newline{}
		``This paper explores the benefits firms may receive from relationship banking that arise from the information gleaned by banks through monitoring. If monitoring reveals not only current output, but also new information about future payoffs, lenders can shield themselves from future losses through early termination of lending agreements. In a competitive lending environment, banks shift the benefits of early termination to borrowers through the lending terms, improving not only the overall expected payoff of projects, but also the welfare of borrowers. Numerical results reveal that the benefits of long-term relationships based on the information revealed in monitoring could be substantial."
		\item \emph{Growth, Income Distribution and Political Economy Implications of Automation} With Manoj Atolia and Jonathan Kreamer. 
		\newline{}
		``We study the distributional consequences of automation in a model with two kinds of agents --- workers, who supply labor, and entrepreneurs, who own capital. We assume that production involves tasks that can be done by either capital or labor with varying productivity. We conceptualize automation as a shift in the relative productivity of capital at certain tasks that reduces the set of tasks done by labor. We contrast this with ``traditional technical progress'', which is an increase in capital productivity at tasks previously done by capital. We derive a simple condition that governs whether labor share goes to zero in the long run, for given tax rates. We then characterize the distributional consequences of a shift in technology, using a tractable case that allows us to cleanly distinguish between automation and traditional technological progress. Finally, we endogenize the tax rate by computing the political economy equilibrium under majority voting, where the government has access to a capital tax and a transfer to workers (a ``universal basic income''). We give conditions for zero or positive capital taxation in the steady state, and conditions under which workers prefer that the labor share go to zero and they derive income wholly from the UBI."
	\end{itemize}}
	

\cvitem{Works in Progress}{This thesis explored the idea that money has been the cause of untold anguish and suffering in the world. I found that it has, in fact, not.}

%----------------------------------------------------------------------------------------
%	WORK EXPERIENCE SECTION
%----------------------------------------------------------------------------------------

\section{Experience}

\subsection{Vocational}

\cventry{2012--Present}{1\textsuperscript{st} Year Analyst}{\textsc{Lehman Brothers}}{Los Angeles}{}{Developed spreadsheets for risk analysis on exotic derivatives on a wide array of commodities (ags, oils, precious and base metals), managed blotter and secondary trades on structured notes, liaised with Middle Office, Sales and Structuring for bookkeeping.
\newline{}\newline{}
Detailed achievements:
\begin{itemize}
\item Learned how to make amazing coffee
\item Finally determined the reason for \textsc{PC LOAD LETTER}:
\begin{itemize}
\item Paper jam
\item Software issues:
\begin{itemize}
\item Word not sending the correct data to printer
\item Windows trying to print in letter format
\end{itemize}
\item Coffee spilled inside printer
\end{itemize}
\item Broke the office record for number of kitten pictures in cubicle
\end{itemize}}

%------------------------------------------------

\cventry{2011--2012}{Summer Intern}{\textsc{Lehman Brothers}}{Los Angeles}{}{Rated "truly distinctive" for Analytical Skills and Teamwork.}

%------------------------------------------------

\subsection{Miscellaneous}

\cventry{2010--2011}{}{}{}{}{Spent some time finding myself. This was a courageous endeavour that didn't have a job title. It was quite important to my overall development though so I'm adding it to my CV. Also it explains the gap in my otherwise stellar CV.}

\cventry{2009--2010}{Computer Repair Specialist}{Buy More}{Burbank}{}{Worked in the Nerd Herd and helped to solve computer problems. Allowed me to become expert in all forms of martial arts and weaponry.}

%----------------------------------------------------------------------------------------
%	AWARDS SECTION
%----------------------------------------------------------------------------------------

\section{Awards}

\cvitem{2011}{School of Business Postgraduate Scholarship}
\cvitem{2010}{Top Achiever Award -- Commerce}

%----------------------------------------------------------------------------------------
%	COMPUTER SKILLS SECTION
%----------------------------------------------------------------------------------------

\section{Computer skills}

\cvitem{Basic}{\textsc{java}, Adobe Illustrator}
\cvitem{Intermediate}{\textsc{python}, \textsc{html}, \LaTeX, OpenOffice, Linux, Microsoft Windows}
\cvitem{Advanced}{Computer Hardware and Support}

%----------------------------------------------------------------------------------------
%	COMMUNICATION SKILLS SECTION
%----------------------------------------------------------------------------------------

\section{Communication Skills}

\cvitem{2010}{Oral Presentation at the California Business Conference}
\cvitem{2009}{Poster at the Annual Business Conference in Oregon}

%----------------------------------------------------------------------------------------
%	LANGUAGES SECTION
%----------------------------------------------------------------------------------------

\section{Languages}

\cvitemwithcomment{English}{Mothertongue}{}
\cvitemwithcomment{Spanish}{Intermediate}{Conversationally fluent}
\cvitemwithcomment{Dutch}{Basic}{Basic words and phrases only}

%----------------------------------------------------------------------------------------
%	INTERESTS SECTION
%----------------------------------------------------------------------------------------

\section{Interests}

\renewcommand{\listitemsymbol}{-~} % Changes the symbol used for lists

\cvlistdoubleitem{Piano}{Chess}
\cvlistdoubleitem{Cooking}{Dancing}
\cvlistitem{Running}

%----------------------------------------------------------------------------------------

\end{document}