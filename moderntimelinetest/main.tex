%%%%%%%%%%%%%%%%%%%%%%%%%%%%%%%%%%%%%%%%%
% "ModernCV" CV and Cover Letter
% LaTeX Template
% Version 1.1 (9/12/12)
%
% This template has been downloaded from:
% http://www.LaTeXTemplates.com
%
% Original author:
% Xavier Danaux (xdanaux@gmail.com)
%
% License:
% CC BY-NC-SA 3.0 (http://creativecommons.org/licenses/by-nc-sa/3.0/)
%
% Important note:
% This template requires the moderncv.cls and .sty files to be in the same 
% directory as this .tex file. These files provide the resume style and themes 
% used for structuring the document.
%
%%%%%%%%%%%%%%%%%%%%%%%%%%%%%%%%%%%%%%%%%

%----------------------------------------------------------------------------------------
%	PACKAGES AND OTHER DOCUMENT CONFIGURATIONS
%----------------------------------------------------------------------------------------

\documentclass[11pt,letterpaper,sans]{moderncv} % Font sizes: 10, 11, or 12; paper sizes: a4paper, letterpaper, a5paper, legalpaper, executivepaper or landscape; font families: sans or roman
\usepackage{standalone}
\moderncvstyle{classic} % CV theme - options include: 'casual' (default), 'classic', 'oldstyle' and 'banking'
\moderncvcolor{black} % CV color - options include: 'blue' (default), 'orange', 'green', 'red', 'purple', 'grey' and 'black'

\usepackage{lipsum} % Used for inserting dummy 'Lorem ipsum' text into the template

\usepackage[scale=0.85]{geometry} % Reduce document margins
%\setlength{\hintscolumnwidth}{3cm} % Uncomment to change the width of the dates column
%\setlength{\makecvtitlenamewidth}{10cm} % For the 'classic' style, uncomment to adjust the width of the space allocated to your name

%\usepackage[utf8]{inputenc}

%\usepackage{booktabs}
\usepackage{fontawesome}
\usepackage{marvosym} % For cool symbols.
%\usepackage{hyperref}

\usepackage[backend=biber, bibstyle = reading, entryhead=false]{biblatex}

\setlength{\bibitemsep}{0.5pt}

\defbibenvironment{bibliography}
{\list
	{}
	{\setlength{\leftmargin}{0pt}%
		\setlength{\itemindent}{0pt}%
		\setlength{\itemsep}{0pt}%
		\setlength{\topsep}{0pt}% 
		\setlength{\parskip}{0pt}%
		\setlength{\partopsep}{0pt}}}
{\endlist}
{\item}

%\defbibenvironment{bibliography}
%{\list
%	{\printtext[labelnumberwidth]{% label format from numeric.bbx
%			\printfield{labelprefix}%
%			\printfield{labelnumber}}}
%	{\setlength{\topsep}{0pt}% layout parameters from moderncvstyleclassic.sty
%		\setlength{\labelwidth}{\hintscolumnwidth}%
%		\setlength{\labelsep}{\separatorcolumnwidth}%
%		\leftmargin\labelwidth%
%		\advance\leftmargin\labelsep}%
%	\sloppy\clubpenalty4000\widowpenalty4000}
%{\endlist}
%{\item}
\addbibresource{jmp.bib}
\addbibresource{wp.bib}
\addbibresource{wip.bib}
\addbibresource{industry.bib}



%----------------------------------------------------------------------------------------
%	NAME AND CONTACT INFORMATION SECTION
%----------------------------------------------------------------------------------------

\firstname{Morgan} % Your first name
\familyname{Holland} % Your last name

% All information in this block is optional, comment out any lines you don't need
\title{Curriculum Vitae}
\address{2750 Old Saint Augustine Rd.}{Tallahassee, FL 32301}
\mobile{(601) 572 7025}
%\phone{(000) 111 1112}
%\fax{(000) 111 1113}
\email{mbholland@fsu.edu}
\homepage{morganbholland.github.io}{morganbholland.github.io} % The first argument is the url for the clickable link, the second argument is the url displayed in the template - this allows special characters to be displayed such as the tilde in this example
%\extrainfo{additional information}
%\photo[70pt][0.4pt]{pictures/picture} % The first bracket is the picture height, the second is the thickness of the frame around the picture (0pt for no frame)
\quote{I am a PhD candidate at Florida State University focused on Financial Economics, Macroeconomics, and Applied Econometrics}




%\social[linkedin][www.linkedin.com]{name}
% The first argument is %the url for the clickable link, the second argument is the url displayed in the %template - this allows special characters to be displayed such as the tilde in this %example

%\photo[70pt][0.3pt]{picture} % The first bracket is the picture height, the second is %the thickness of the frame around the picture (0pt for no frame)
%\quote{Not Attention, Patience is all we need.}

%----------------------------------------------------------------------------------------

\newcommand{\cvdoublecolumn}[2]{%
  \cvitem[.75em]{}{%
    \begin{minipage}[t]{\listdoubleitemcolumnwidth}#1\end{minipage}%
    \hfill%
    \begin{minipage}[t]{\listdoubleitemcolumnwidth}#2\end{minipage}%
    }%
}



%\usepackage{multibbl}


\usepackage[firstyear=2007,lastyear=2021,
marksyr=0.4ex,marksmo=0.2ex
]{moderntimeline}
\tltextstart[base]{\scriptsize}
\tltextend[north]{\scriptsize}
\tldisablemarksyr
% \newcommand{\cvreference}[7]{%
%     \textbf{#1}\newline% Name
%     \ifthenelse{\equal{#2}{}}{}{\addresssymbol~#2\newline}%
%     \ifthenelse{\equal{#3}{}}{}{#3\newline}%
%     \ifthenelse{\equal{#4}{}}{}{#4\newline}%
%     \ifthenelse{\equal{#5}{}}{}{#5\newline}%
%     \ifthenelse{\equal{#6}{}}{}{\emailsymbol~\texttt{#6}\newline}%
%     \ifthenelse{\equal{#7}{}}{}{\phonesymbol~#7}}



\begin{document}
	
	
%----------------------------------------------------------------------------------------
%	COVER LETTER
%----------------------------------------------------------------------------------------

% To remove the cover letter, comment out this entire block

%\clearpage
%
%\recipient{HR Department}{Corporation\\123 Pleasant Lane\\12345 City, State} % Letter recipient
%\date{\today} % Letter date
%\opening{Dear Sir or Madam,} % Opening greeting
%\closing{Sincerely yours,} % Closing phrase
%\enclosure[Attached]{curriculum vit\ae{}} % List of enclosed documents
%
%\makelettertitle % Print letter title
%
%\lipsum[1-2] % Dummy text
%\lipsum[4] % Dummy text
%
%\makeletterclosing % Print letter signature
%
%\newpage

\makecvtitle % Print the CV title




%----------------------------------------------------------------------------------------
%	EDUCATION SECTION
%----------------------------------------------------------------------------------------



\section{Education}

\cventry{May 2022}{Doctorate of Philosophy in Economics}{Florida State University COSSPP (Anticipated)}{}{}{} % Arguments not required can be left empty
\cventry{Dec 2014}{Master of Arts in Economics}{The University of South Carolina Moore School of Business}{}{}{}
\cventry{Dec 2010}{Bachelor of Business Administration}{Mississippi College School of Business}{}{}{}
\cventry{May 2009}{Associate of Arts in Business Administration}{Meridian Community College}{}{}{}





%----------------------------------------------------------------------------------------
%	PUBLICATION SECTION
%----------------------------------------------------------------------------------------
\section{Research}

\cvitem{Job Market Paper}{\textbf{Morgan Holland}. \emph{The Role of Long-term and Short-term Risk in Relationship Banking}.\newline%
Abstract: This paper explores the benefits firms may receive from relationship banking that arise from the information gleaned by banks through monitoring. If monitoring reveals not only current output, but also new information about future payoffs, lenders can shield themselves from future losses through early termination of lending agreements. In a competitive lending environment, banks shift the benefits of early termination to borrowers through the lending terms, improving not only the overall expected payoff of projects, but also the welfare of borrowers. Numerical results reveal that the benefits of long-term relationships based on the information revealed in monitoring could be substantial.}

\cvitem{Working Paper}{Manoj Atolia, \textbf{Morgan Holland}, and Jonathan Kreamer. \emph{The Role of Long-term and Short-term Risk in Relationship Banking}.\newline%
	Abstract: We study the distributional consequences of automation in a model with two kinds of agents --- workers, who supply labor, and entrepreneurs, who own capital. We assume that production involves tasks that can be done by either capital or labor with varying productivity. We conceptualize automation as a shift in the relative productivity of capital at certain tasks that reduces the set of tasks done by labor. We contrast this with ``traditional technical progress'', which is an increase in capital productivity at tasks previously done by capital. We derive a simple condition that governs whether labor share goes to zero in the long run, for given tax rates. We then characterize the distributional consequences of a shift in technology, using a tractable case that allows us to cleanly distinguish between automation and traditional technological progress. Finally, we endogenize the tax rate by computing the political economy equilibrium under majority voting, where the government has access to a capital tax and a transfer to workers (a ``universal basic income''). We give conditions for zero or positive capital taxation in the steady state, and conditions under which workers prefer that the labor share go to zero and they derive income wholly from the UBI.}

\cvitem{Works in Progress}{\textbf{Morgan Holland} \emph{Wage, Wealth, and Income Inequality from Automation}.\newline
Abstract: There are three distinct kinds of inequality affected by automation. \emph{Wage} inequality arises as the jobs performed by some workers are substituted by capital, while other jobs are created by, and complementary with capital. \emph{Income} inequality comes from not only wage inequality, but also higher returns to capital when automation occurs. Exacerbating income inequality is \emph{Wealth} inequality, where those with higher wages and income are able to invest in automation capital at higher rates. I build a task-based model of automation that incorporates all three kinds of inequality. I match this model to U. S. data to determine the impact automation has had on inequality and to predict how future automation could affect inequality in the future. Finally, I explore policy that could reduce inequality from automation.\newline%
%\vspace{\baselineskip}%
\newline
Margaret Holland and \textbf{Morgan Holland} \emph{The Impact of the Americans with Disabilities Act on the Employment and Earnings of People with Disabilities: An Intersectional Approach}.\newline
The Americans with Disabilities Act (ADA) was designed to decrease discrimination by employers against people with disabilities and thereby improve the earnings and employment of people with disabilities. Motivated by intersectionality, we use difference in differences to determine how the ADA affected disparities in employment and earnings differently for marginalized populations.}

%\begin{refsection}
%	\nocite{holland2021role}%		
%\cvline{JMP}{%
%	%\vspace{-\parskip}
%	\printbibliography[heading=none]%		
%		%\bibliographystyle{reading}
%		}
%\end{refsection}
%
%
%\subsection{Working Paper}
%\begin{refsection}
%\nocite{atolia2021growth}
%\printbibliography[segment=2, heading=none]
%\end{refsection}
%
%\subsection{Works In Progress}
%\begin{refsection}
%	\nocite{holland2021wage,holland2021impact}
%	\printbibliography[segment=3, heading=none]
%\end{refsection}
%
%\subsection{Industry and Consulting}
%\begin{refsection}
%	\nocite{holland2021florida, holland2021humana}
%	\printbibliography[segment=4, heading=none]
%\end{refsection}


\section{Teaching}
\subsection{Instructor}
\cventry{2016--2021}{Analysis of Economic Data}{}{}{}{Upper-level course leading into Introduction to Econometrics. 
	\begin{itemize}
		\item Developed custom instruction materials for statistics focusing on economic data.
		\item Integrated R programming to teach students practical data management and analytical tools.
		\item Engaged with students using real-world examples of using data analysis and econometrics to answer economic questions.
		\item Topics:
		\begin{itemize}
			\item Data cleaning and preparation
			\item Probability
			\item Statistical hypothesis testing
			\item Linear regression
		\end{itemize}
	\end{itemize}}

\cventry{2017--200}{Introduction to Econometrics}{}{}{}{
	\begin{itemize}
		\item Integrated R programming and econometric theory in the classroom and in online environements
		\item Topics
			\begin{itemize}
				\item Linear regression for causal analysis
				\item Gauss-Markov Theory and applications
				\item Extensions and modifications for failures of the Gauss-Markov assumptions
				\item Real-world examples of econometric analysis, its successes and limitations
			\end{itemize}
\end{itemize}}

\subsection{Teaching Assistant}
\cvitem{}{\textbf{Financial Markets, Banking, and Monetary Policy}}
\cvitem{}{\textbf{Economics of Population}}



%----------------------------------------------------------------------------------------
%	WORK EXPERIENCE SECTION
%----------------------------------------------------------------------------------------

\section{Professional Experience}

\cventry{2021--}{Senior Researcher}{Florida State University Center for Economic Forecasting and Analysis}{Tallahassee}{}{\begin{itemize}
		\item Communicated with clients and partners to define the scope of economic questions  and resources needed for analysis.
		\item Collected, organized, and prepared data for analysis
		\item Gathered and summarized literature relevant to the context and methodology of client problems
		\item Performed economic and econometric analysis to answer questions and provide solutions for clients and partners
\end{itemize}}

\cventry{2013}{Graduate Research Assistant}{University of South Carolina, Moore School of Business}{Columbia}{}{\begin{itemize}
		\item Located and retrieved ownership data from Initial Public Offering (IPO) prospectuses.
		\item Constructed custom spreadsheets of IPO data.
		\item Performed literature searches for scholarly articles on several subjects related to corporate finance.
\end{itemize}}

\cventry{2012--2013}{Auditor}{South Carolina Office of Regulatory Staff}{Columbia}{}{\begin{itemize}
		\item Advocated on behalf of the public in utility rate case proceedings before the South Carolina Public Service Commission.
		\item Audited utility rate case filings to verify the accuracy of financial data.
		\item Recommended adjustments to test-year revenue requirement analyses based on company financial data and rate case filings.
		\item Crafted written testimony to be filed with the South Carolina Public Service Commission.
		\item Reviewed nuclear plant construction invoices for compliance with the South Carolina Base Load Review Act and consistency with company financial documents.
		\item Examined telecommunications company filings and financial documents for compliance with telecommunications law.
\end{itemize}}

\cventry{2011--2012}{Accountant/Auditor}{Mississippi Department of Finance and Administration}{Jackson}{}{\begin{itemize}
		\item Pre-audited payments to State Treasurer for errors.
		\item Assisted state agencies in completing payments to the State Treasury.
		\item Approved transactions in a computerized accounting system.
		\item Verified accuracy of daily deposits to the State Treasury.
\end{itemize}}

\section{Programming and Software}

\cvitem{Expert}{R, Julia, Excel, Word, PowerPoint}

\cvitem{Intermediate}{Matlab, Stata}

\cvitem{Beginner}{Python, MySQL, MongoDB}

%----------------------------------------------------------------------------------------
%	Fellowships \& Awards
%----------------------------------------------------------------------------------------

\section{Fellowships, Memberships, Miscellaneous}

\cvitem{2015--2018}{Johnson Fellowship, Florida State University}
\cvitem{2013--2015}{Merit Scholarship, Moore School of Business}
\cvitem{2009-2010}{Phi Theta Kappa Alumni Scholarship. Mississippi College}
\cvitem{2007-2009}{Phi Theta Kappa member. Meridian Community College}
\cvitem{}{PADI certified rescue diver}

%----------------------------------------------------------------------------------------
%	Academic achievements
%----------------------------------------------------------------------------------------


%----------------------------------------------------------------------------------------
%	COMPUTER SKILLS SECTION
%----------------------------------------------------------------------------------------


%----------------------------------------------------------------------------------------
%	Position of Responsibility SECTION
%----------------------------------------------------------------------------------------

%----------------------------------------------------------------------------------------
%	Teaching Assistantship SECTION
%----------------------------------------------------------------------------------------

%\section{Referees}
%
%
%\begin{tabular}{lr}
%% Referee 1
%\begin{minipage}[t]{3in}
%\textbf{Dr. XXXXX XXXXX}\\
%\textit{Associate Professor, Department of} \\
%\textit{Computer Science \& Engineering}\\
%Institute name\\
%\Letter\ \href{mailto:abc@gmail.com}{abc@gmail.com}
%\end{minipage}
%&
%% Referee 2
%\begin{minipage}[t]{3in}
%\textbf{Dr. XXXXX XXXXX}\\
%\textit{Associate Professor, Department of} \\
%\textit{Computer Science \& Engineering}\\
%Institute name\\
%\Telefon\ +(601) 877-6236\\
%\Letter\ \href{mailto:abc@gmail.com}{abc@gmail.com}
%\end{minipage}
%\\
%\\ % Additional newline for spacing.
%% Referee 3
%\begin{minipage}[t]{3in}
%\textbf{Dr. XXXXX XXXXX}\\
%\textit{Associate Professor, Department of} \\
%\textit{Computer Science \& Engineering}\\
%Institute name\\
%\Telefon\ +(601) 877-6236\\
%\Letter\ \href{mailto:abc@gmail.com}{abc@gmail.com}
%\end{minipage}
%&
%\\
%\end{tabular}


\end{document}