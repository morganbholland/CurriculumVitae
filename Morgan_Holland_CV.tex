%%%%%%%%%%%%%%%%%%%%%%%%%%%%%%%%%%%%%%%%%
% Medium Length Professional CV
% LaTeX Template
% Version 2.0 (8/5/13)
%
% This template has been downloaded from:
% http://www.LaTeXTemplates.com
%
% Original author:
% Trey Hunner (http://www.treyhunner.com/)
%
% Important note:
% This template requires the resume.cls file to be in the same directory as the
% .tex file. The resume.cls file provides the resume style used for structuring the
% document.
%
%%%%%%%%%%%%%%%%%%%%%%%%%%%%%%%%%%%%%%%%%

%----------------------------------------------------------------------------------------
%	PACKAGES AND OTHER DOCUMENT CONFIGURATIONS
%----------------------------------------------------------------------------------------

\documentclass{resume} % Use the custom resume.cls style

\usepackage[left=0.75in,top=0.6in,right=0.75in,bottom=0.6in]{geometry} % Document margins
\usepackage{hyperref} % For linking working papers, etc.
\hypersetup{
	colorlinks=true,
	linkcolor=blue,
	filecolor=magenta,      
	urlcolor=red,
	pdftitle={CV},
	pdfpagemode=FullScreen
}

\usepackage{enumitem}
\setlist{nosep}
\setlist[itemize,1]{label=$*$}

\name{Morgan Holland} % Your name
\address{601-572-7025 \\ morganbholland@gmail.com \\ 2750 Old Saint Augustine Rd. D33 Tallahassee, FL 32301} % Your phone number and email

\begin{document}

%----------------------------------------------------------------------------------------
%	EDUCATION SECTION
%----------------------------------------------------------------------------------------

\begin{rSection}{Education}

{\bf Florida State University, COSSPP} \hfill {\em 2022 (expected)} \\ 
Doctorate of Philosophy in Economics \\

{\bf University of South Carolina, Moore School of Business} \hfill {\em 2015} \\
Master of Arts in Economics \\

{\bf Mississippi College, School of Business} \hfill {\em 2010}\\
Bachelor of Science in Business Administration

\end{rSection}

%----------------------------------------------------------------------------------------
%	RESEARCH EXPERIENCE SECTION
%----------------------------------------------------------------------------------------

\begin{rSection}{Academic Research}
	    \begin{rSubsection}{Job Market Paper}{}{}{}
		\item \emph{The Role of Long-Term and Short-Term Risk in Relationship Banking}.\\
		Abstract: ``This paper explores the benefits firms may receive from relationship banking that arise from the information gleaned by banks through monitoring. If monitoring reveals not only current output, but also new information about future payoffs, lenders can shield themselves from future losses through early termination of lending agreements. In a competitive lending environment, banks shift the benefits of early termination to borrowers through the lending terms, improving not only the overall expected payoff of projects, but also the welfare of borrowers. Numerical results reveal that the benefits of long-term relationships based on the information revealed in monitoring could be substantial."
		\end{rSubsection}
	
		\begin{rSubsection}{Working Papers}{}{}{}

		\item \emph{Growth, Income Distribution and Political Economy Implications of Automation}. With Manoj Atolia and Jonathan Kreamer\\
		Abstract: ``We study the distributional consequences of automation in a model with two kinds of agents --- workers, who supply labor, and entrepreneurs, who own capital. We assume that production involves tasks that can be done by either capital or labor with varying productivity. We conceptualize automation as a shift in the relative productivity of capital at certain tasks that reduces the set of tasks done by labor. We contrast this with ``traditional technical progress'', which is an increase in capital productivity at tasks previously done by capital. We derive a simple condition that governs whether labor share goes to zero in the long run, for given tax rates. We then characterize the distributional consequences of a shift in technology, using a tractable case that allows us to cleanly distinguish between automation and traditional technological progress. Finally, we endogenize the tax rate by computing the political economy equilibrium under majority voting, where the government has access to a capital tax and a transfer to workers (a ``universal basic income''). We give conditions for zero or positive capital taxation in the steady state, and conditions under which workers prefer that the labor share go to zero and they derive income wholly from the UBI." \\
	\end{rSubsection}
	
    \begin{rSubsection}{Works in Progress}{}{}{}
        \item \emph{Wage, Wealth, and Income Inequality from Automation}\\
        Working Abstract: ``There are three distinct kinds of inequality affected by automation. \emph{Wage} inequality arises as the jobs performed by some workers are substituted by capital, while other jobs are created by, and complementary with capital. \emph{Income} inequality comes from not only wage inequality, but also higher returns to capital when automation occurs. Exacerbating income inequality is \emph{Wealth} inequality, where those with higher wages and income are able to invest in automation capital at higher rates. I build a task-based model of automation that incorporates all three kinds of inequality. I match this model to U. S. data to determine the impact automation has had on inequality and to predict how future automation could affect inequality in the future. Finally, I explore policy that could reduce inequality from automation."
        \item \emph{The Impact of the Americans with Disabilities Act on the Employment and Earnings of People with Disabilities: An Intersectional Approach}. With Margaret Holland\\
        Working Abstract: ``The Americans with Disabilities Act (ADA) was designed to decrease discrimination by employers against people with disabilities and thereby improve the earnings and employment of people with disabilities. Motivated by intersectionality, we use difference in differences to determine how the ADA affected disparities in employment and earnings differently for marginalized populations."
        
    \end{rSubsection}

\end{rSection}


%----------------------------------------------------------------------------------------
%	TEACHING EXPERIENCE SECTION
%----------------------------------------------------------------------------------------
\begin{rSection}{Teaching}
    \begin{rSubsection}{Florida State University}{6 semesters}{Instructor, Analysis of Economic Data, Econometrics}{Tallahassee, FL}
        \item Developed custom instruction materials for statistics focusing on economic data and econometrics.
        \item Integrated R programming to teach students practical data management and analytical tools.
        \item Engaged with students using real-world examples of using data analysis and econometrics to answer economic questions.
        \item Covered subjects spanning from introductory statistics topics to advanced topics including
        
        \begin{itemize}
            \item Graphical and numerical techniques for summarizing data,
            \item Cleaning and preparing data for analysis,
            \item Probability theory and hypothesis testing,
            \item Linear regression for prediction and causal analysis,
            \item Panel data and methods,
            \item Generating predictions from time series data.
        \end{itemize}
    \end{rSubsection}
\end{rSection}


\begin{rSection}{Industry, Consulting, and Miscellaneous Research}
	\item \href{https://economic-impact.fsu.edu/}{{\em Florida State's Economic Impact}}. 2021 version based on 2020 data.
	\item {\em The Impact of Humana's Operations on Florida's Economy} With Julie Harrington (in progress)
	
\end{rSection}

\begin{rSection}{Professional Experience}
    
    %------------------------------------------------

%------------------------------------------------
\begin{rSubsection}{Florida State University Center for Economic Forecasting and Analysis}{September 2021 - }{Senior Researcher}{Tallahassee, FL}
	\item Communicated with clients and partners to define the scope of economic questions  and resources needed for analysis.
	\item Collected, organized, and prepared data for analysis
	\item Gathered and summarized literature relevant to the context and methodology of client problems
	\item Performed economic and econometric analysis to answer questions and provide solutions for clients and partners
\end{rSubsection}


\begin{rSubsection}{University of South Carolina}{September 2013 - December 2013}{Research Assistant}{Columbia, SC}
    \item Located and retrieved ownership data from Initial Public Offering (IPO) prospectuses.
    \item Constructed custom spreadsheets of IPO data.
    \item Performed literature searches for scholarly articles on several subjects related to corporate finance.
\end{rSubsection}
%------------------------------------------------

\begin{rSubsection}{South Carolina Office of Regulatory Staff}{July 2012 - September 2013}{Auditor}{Columbia, SC}
    \item Advocated on behalf of the public in utility rate case proceedings before the South Carolina Public Service Commission.
    \item Audited utility rate case filings to verify the accuracy of financial data.
    \item Recommended adjustments to test-year revenue requirement analyses based on company financial data and rate case filings.
    \item Crafted written testimony to be filed with the South Carolina Public Service Commission.
    \item Reviewed nuclear plant construction invoices for compliance with the South Carolina Base Load Review Act and consistency with company financial documents.
    \item Examined telecommunications company filings and financial documents for compliance with telecommunications law.

\end{rSubsection}
%------------------------------------------------

\begin{rSubsection}{Mississippi Department of Finance and Administration}{January 2011-July 2012}{Accountant/Auditor}{Jackson, MS}
    \item Pre-audited payments to State Treasurer for errors.
    \item Assisted state agencies in completing payments to the State Treasury.
    \item Approved transactions in a computerized accounting system.
    \item Verified accuracy of daily deposits to the State Treasury.

\end{rSubsection}
\end{rSection}

%----------------------------------------------------------------------------------------
%	COMPUTER PROFICIENCIES SECTION
%----------------------------------------------------------------------------------------

\begin{rSection}{Software Proficiency}

\begin{rSubsection}{Programming}{}{}{}
    \item {\em Expert}: R, Julia
    \item {\em Intermediate}: Matlab, Stata 
    \item {\em Beginner}: Python
\end{rSubsection}

\begin{rSubsection}{Spreadsheet and Database}{}{}{}
    \item {\em Intermediate}: Excel 
    \item {\em Beginner}: SQL, MongoDB
\end{rSubsection}

\begin{rSubsection}{Document and Presentation Creation, Literate Programming}{}{}{}
    \item \LaTeX, Lyx, R Sweave, R Markdown, Jupyter, Word, Powerpoint
\end{rSubsection}

\begin{rSubsection}{Version Control}{}{}{}
    \item GitHub
\end{rSubsection}

\end{rSection}

%----------------------------------------------------------------------------------------
%	Academic Research
%----------------------------------------------------------------------------------------



\begin{rSection}{Fellowships, Memberships, Miscellaneous}
	\item Johnson Fellowship. Florida State University
	\item Individual Fellowship. Moore School of Business
	\item Phi Theta Kappa Alumni Scholarship. Mississippi College
	\item Phi Theta Kappa. 2-year college honor society
	\item PADI certified rescue diver
\end{rSection}

%----------------------------------------------------------------------------------------
%	Miscellaneous 
%----------------------------------------------------------------------------------------

% \begin{rSection}{}

% Section content\ldots

% \end{rSection}

%----------------------------------------------------------------------------------------

\end{document}
