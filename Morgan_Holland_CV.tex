%%%%%%%%%%%%%%%%%%%%%%%%%%%%%%%%%%%%%%%%%
% Medium Length Professional CV
% LaTeX Template
% Version 2.0 (8/5/13)
%
% This template has been downloaded from:
% http://www.LaTeXTemplates.com
%
% Original author:
% Trey Hunner (http://www.treyhunner.com/)
%
% Important note:
% This template requires the resume.cls file to be in the same directory as the
% .tex file. The resume.cls file provides the resume style used for structuring the
% document.
%
%%%%%%%%%%%%%%%%%%%%%%%%%%%%%%%%%%%%%%%%%

%----------------------------------------------------------------------------------------
%	PACKAGES AND OTHER DOCUMENT CONFIGURATIONS
%----------------------------------------------------------------------------------------

\documentclass{resume} % Use the custom resume.cls style

\usepackage[left=0.75in,top=0.6in,right=0.75in,bottom=0.6in]{geometry} % Document margins

\usepackage{enumitem}
\setlist{nosep}
\setlist[itemize,1]{label=$*$}

\name{Morgan Holland} % Your name
\address{601-572-7025 \\ morganbholland@gmail.com \\ 2750 Old Saint Augustine Rd. D33 Tallahassee, FL 32301} % Your phone number and email

\begin{document}

%----------------------------------------------------------------------------------------
%	EDUCATION SECTION
%----------------------------------------------------------------------------------------

\begin{rSection}{Education}

{\bf Florida State University, COSSPP} \hfill {\em 2022 (expected)} \\ 
Doctorate of Philosophy in Economics \\

{\bf University of South Carolina, Moore School of Business} \hfill {\em 2015} \\
Master of Arts in Economics \\

{\bf Mississippi College, School of Business} \hfill {\em 2010}\\
Bachelor of Science in Business Administration

\end{rSection}

%----------------------------------------------------------------------------------------
%	RESEARCH EXPERIENCE SECTION
%----------------------------------------------------------------------------------------

\begin{rSection}{Research Experience}
    \begin{rSubsection}{Economics of Automation}{}{}{}
        \item Crafted literature reviews on the economics of automation, substitution between labor and capital in economic models, and optimal capital taxation
        \item Developed task-based growth models investigating the relationship between automation, political economy, and economic growth and inequality.
        \item Investigated the relationship between automation and real business cycles using a dynamic stochastic general equilibrium (DSGE) model.
        \item Used log-linearization and value function iteration solution methods for DSGE models
    \end{rSubsection}

    \begin{rSubsection}{The Option Value of Intermediate Payments in Relationship Banking}{}{}{}
        \item Developed a multiperiod optimal debt model to investigate the role of intermediate payments
        \item Employed numerical techniques to solve the model, including 
        \begin{itemize}
            \item adaptive Gauss-Kronrod and adaptive cubature algorithms for numerical integration
            \item Newton with Trust Region and BFGS algorithms for maximizing continuous functions
            \item Nelder-Mead algorithm for maximizing discontinuous functions
        \end{itemize}
    \end{rSubsection}

    \begin{rSubsection}{Other Research}{}{}{}
        \item Investigated trends and links between availability and use of credit and real business cycles using multivariate time-series methods.
        \item Analyzed factors contributing to recidivism in South Carolina prisons using survival analysis methods.
    \end{rSubsection}
\end{rSection}

%----------------------------------------------------------------------------------------
%	TEACHING EXPERIENCE SECTION
%----------------------------------------------------------------------------------------
\begin{rSection}{Teaching Experience}
    \begin{rSubsection}{Analysis of Economic Data}{4 semesters}{}{}
        \item Developed custom instruction materials for statistics focusing on economic data and preparation for Introduction to Econometrics
        \item Integrated R programming to teach students data management and statistical analysis.
        \item Taught fundamental knowledge needed to understand data, including probability theory, graphical and sample techniqes for generating hypotheses, hypothesis testing using sample means, and using linear regression. 
        \item Engaged with students using real-world examples of using data analysis to answer economic questions
    \end{rSubsection}
    \begin{rSubsection}{Introduction to Econometrics}{2 semesters}{}{}
        \item Introduced students to using linear regression in econometric analysis
        \item Taught hypothesis testing in linear regression, including the fundamental assumptions needed for causality
        \item Explored alternative methods to ameliorate violations of the linear regression assumptions
        \item 
        
    \end{rSubsection}
\end{rSection}
%----------------------------------------------------------------------------------------
%	WORK EXPERIENCE SECTION
%----------------------------------------------------------------------------------------

\begin{rSection}{Work Experience}



\begin{rSubsection}{Florida State University}{May 2017 - June 2021}{Instructor/Teaching Assistant}{Tallahassee, FL}
    \item Taught undergraduate level statistics and econometrics in person and online. 
    \item Integrated statistical software with course materials.
    \item Developed custom instruction materials for statistics focusing on economic data and econometric analysis.
    \item Assisted faculty in developing course materials, class management, and grading.

\end{rSubsection}

\begin{rSubsection}{Florida State University}{August 2015 - May 2019}{Research Assistant}{Tallahassee, FL}
    \item Proposed and developed economic research projects both individually and in collaboration with faculty.
    \item Compiled, cleaned, and analyzed data from Federal and international agencies to generate and investigate research hypotheses. Data analysis methods included
    \begin{itemize}
        \item Single and multivariate time series methods for detecting trends and relationships between economic indicators
        \item Panel methods for causal analysis, including fixed effects and difference-in-differences
        \item Nonlinear methods including limited dependent variables (i.e Tobit, Probit, logistic regression) and hazard modelling (nonproportional Weibull models)
    \end{itemize}
    \item Developed mathematical economic models for research projects investigating
    \begin{itemize}
        \item The relationship between 
        \item The relationship between Real Business Cycles and automation.
        \item Moral hazard and monitoring in relationship banking.
    \end{itemize}
    \item Employed numerical techniques to solve economic modelling problems. Methods include 
    \begin{itemize}
        \item Adaptive Gauss-Kronrod quadrature for single-variable numerical integration, 
        \item Adaptive cubature for multivariate integration
        \item Newton with trust region, BFGS algorithms for local optimization of continuous functions
        \item Nelder-Mead algorithm for local optimization of discontinuous functions 
        \item Log-linearization and value function iteration techniques for solving dynamic stochastic general equilibrium models
    \end{itemize} 
    
\end{rSubsection}
    
    %------------------------------------------------

%------------------------------------------------

\begin{rSubsection}{University of South Carolina}{September 2013 - December 2013}{Research Assistant}{Columbia, SC}
    \item Located and retrieved ownership data from Initial Public Offering (IPO) prospectuses.
    \item Constructed custom spreadsheets of IPO data.
    \item Performed literature searches for scholarly articles on several subjects related to corporate finance.
\end{rSubsection}
%------------------------------------------------

\begin{rSubsection}{South Carolina Office of Regulatory Staff}{July 2012 - September 2013}{Auditor}{Columbia, SC}
    \item Advocated on behalf of the public in utility rate case proceedings before the South Carolina Public Service Commission.
    \item Audited utility rate case filings to verify the accuracy of financial data.
    \item Recommended adjustments to test-year revenue requirement analyses based on company financial data and rate case filings.
    \item Crafted written testimony to be filed with the South Carolina Public Service Commission.
    \item Reviewed nuclear plant construction invoices for compliance with the South Carolina Base Load Review Act and consistency with company financial documents.
    \item Examined telecommunications company filings and financial documents for compliance with telecommunications law.

\end{rSubsection}
%------------------------------------------------

\begin{rSubsection}{Mississippi Department of Finance and Administration}{January 2011-July 2012}{Accountant/Auditor}{Jackson, MS}
    \item Pre-audited payments to State Treasurer for errors.
    \item Assisted state agencies in completing payments to the State Treasury.
    \item Approved transactions in a computerized accounting system.
    \item Verified accuracy of daily deposits to the State Treasury.


\end{rSubsection}
\end{rSection}

%----------------------------------------------------------------------------------------
%	COMPUTER PROFICIENCIES SECTION
%----------------------------------------------------------------------------------------

\begin{rSection}{Computer Proficiencies}

\begin{rSubsection}{Programming}{}{}{}
    \item {\em Expert}: R, Julia
    \item {\em Intermediate}: Matlab, Stata 
    \item {\em Beginner}: Python (Numpy, Pandas)
\end{rSubsection}

\begin{rSubsection}{Spreadsheet and Database}{}{}{}
    \item {\em Intermediate}: Excel 
    \item {\em Beginner}: SQL, MongoDB, JSON
\end{rSubsection}

\begin{rSubsection}{Document and Presentation Creation, Literate Programming}{}{}{}
    \item \LaTeX, Lyx, Rweave, Rmarkdown, Jupyter, Word, Powerpoint
\end{rSubsection}

\begin{rSubsection}{Version Control}{}{}{}
    \item GitHub
\end{rSubsection}

\end{rSection}

%----------------------------------------------------------------------------------------
%	Academic Research
%----------------------------------------------------------------------------------------
\begin{rSection}{Academic Research}

\begin{rSubsection}{Working Papers}{}{}{}
    \item {\em Growth, Income Distribution, and Political Economy Implications of Automation} (With Manoj Atolia and Jonathan Kreamer)
    \item {\em The Option Value of Monitoring in Relationship Banking}
\end{rSubsection}

\begin{rSubsection}{Works in Progress}{}{}{}
    \item {\em The Effect of the Americans with Disabilities Act on Earnings and Employment Outcomes for People with Disabilities: An Intersectional Approach}
\end{rSubsection}

\end{rSection}

%----------------------------------------------------------------------------------------
%	Miscellaneous 
%----------------------------------------------------------------------------------------

% \begin{rSection}{}

% Section content\ldots

% \end{rSection}

%----------------------------------------------------------------------------------------

\end{document}
